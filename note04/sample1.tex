\documentclass[fontsize=12pt, paper=a4]{jlreq}

%以降行末まではコメント。

%%%%%%%%%%%%%%%%%%%%%%%%%%%%%%%%%%%%%%%%%%%%%%%%%%%%%%%%%%%%%%%%%%%%%%%%
%\usepackage{graphicx} %%% 図を取り込むときに使う


%%%%%%%%%%%%%%%%%%%%%%%%%%%%%%%%%%%%%%%%%%%%%%%%%%%%%%%%%%%%%%%%%%%%%%%%
%% タイトル
%%%%%%%%%%%%%%%%%%%%%%%%%%%%%%%%%%%%%%%%%%%%%%%%%%%%%%%%%%%%%%%%%%%%%%%%
%% 題目
\title{\LaTeX での文章の書き方の基礎}
%% 著者名
\author{筑波大学 三末和男 (改訂:中井央)}
\date{2013年5月4日}
%% 日付: 指定しなければ、今日の日付が入る。

%% 文章開始
\begin{document}

%%%%%%%%%%%%%%%%%%%%%%%%%%%%%%%%%%%%%%%%%%%%%%%%%%%%%%%%%%%%%%%%%%%%%%%%
%% タイトル作成
%%%%%%%%%%%%%%%%%%%%%%%%%%%%%%%%%%%%%%%%%%%%%%%%%%%%%%%%%%%%%%%%%%%%%%%%
\maketitle

%%%\chapter{はじめに}


\section{はじめに}


\LaTeX の特長の一つは、文章の見た目ではなく構造を記述することにある。
たとえば、「はじめに」は章\footnote{英文の場合、
  このくらいの長さの文章には「章(chapter)」という単位は使わない。
  このくらいの長さの文章(article)では、最も大きい単位は、
  「節(section)」である。
  ただし、日本語ではこのくらいの文章でも「章」と呼ぶことは多い。
}
の見出しだから大きめの16ポイントで表示しようと
ワードプロセッサに指示するのではなく、
「はじめに」は「章の見出しである」ということを記述する。
つまり論理的な構造と見た目を分けて管理できるということである。
このことは様々な利点がある。
特に長い文章を書く時などに重宝する。

\section{文書クラス}\label{sec:documentclass}

文章を論理構造に着目して記述するために、
まずどのような種類の文書かを指定しておく必要がある。
\LaTeX では下のような文書クラスが用意されている。

\begin{description}
\item[book] 本 (jbook で日本語の本、tbook で縦書きの本。以下同様)
\item[article] 学術論文
\item[report] レポート
%\item[letter] 手紙
\end{description}


これらはソースファイルの先頭で次のように指定する。

{\small
\begin{verbatim}
\documentclass[fontsize=12pt, paper=a4]{jlreq}
\end{verbatim}
}

\verb+jlreq+ を指定することで article 相当になっている。
book, report を使用したい場合は、\verb+[+ と \verb+]+ の間にそれを記述する。


\section{章立て}

文章は、章(chapter)、節(section)、小節(section)、
段落(paragraph)などによって構成される。
それぞれの表題を次のように記述する。

\begin{quote}
\begin{verbatim}
\section{ここに表題を書く}
\end{verbatim}
\end{quote}

この節のタイトルは\verb|\section{\LaTeX における文書構造}|
と記述している。
同様にこの小節は\verb|\section{章立て}|のように記述している。
章や節の番号を書く必要はない、番号は自動的に付与される。
あえて番号を付けたくない場合には、次のように\verb|*|を付けて記述する。

\begin{quote}
\begin{verbatim}
\section*{番号を付けない節}
\end{verbatim}
\end{quote}


\section{参照}

「第2節で述べたように」と前や後の章や節を参照したい場合がある。
通例、このような記述は3節以降に現れるが、
もし構成を変更して、1節と2節の間に新しい節を設けると、
すべての「第2節」を「第3節」に書き換える必要が出てくる。
さらに、やっぱり、新しく入れた
2節を1節の最後の小節にすることにしたら、
同じようにしてすべての「第3節」を変更する必要がある。

このような場合には参照名を使用する。参照名は
\begin{quote}
\begin{verbatim}
\section{参照される節}
\label{sec:sansyo-sareru}
\end{verbatim}
\end{quote}
のように章や節の表題の直後で\verb|\label{...}|を用いて定義する。
これを参照する場合には、\verb|\ref{...}|を用いる。
\begin{quote}
\begin{verbatim}
先に第\ref{sec:sansyo-sareru}節で説明したように、
\end{verbatim}
\end{quote}
と記述することで、節の番号が自動的に埋め込まれる。

参照は前方でも後方でも構わない。
どちらの場合でも、
参照を使用した場合には2回コンパイルする必要があることに注意すること。
1回目のコンパイルでは参照個所が
\begin{quote}
先に第??節で説明したよう、に
\end{quote}
のようになってしまう。

参照は、章や節の番号以外でも、
箇条書きや数式、図、表(後で学習する)でも使用できる。
参考文献についても類似の記法で文献番号を使用する。


\section{文の書き方}\label{writingSentences.tex}

文は基本的には通常の文字列として記述する。
ただし、いくつかの注意事項がある。
以下の文字(半角文字)は特別な意味を持つためそのままでは出力されない。

\begin{quote}
\begin{verbatim}
# $ % & _ { } \ ^ ~
\end{verbatim}
\end{quote}

たとえば、\verb|%|以降行末まではコメントとみなされ無視される。
% 古い LaTeX では下記2行のようなことが起こっていた。
%うっかり不等号の記号\verb|<|や\verb|>|を文中に埋め込む失敗がしばしばある。
%これは「<」と「>」に化けてしまう。

文字と文字の間に          どんなにをたくさん空白を入れても1つの空白 
(英単語間の間隔) として扱われる。
スペースを空けたい場合にはそのための命令を使用する必要がある
\footnote{全角文字の空白はそのまま空白として扱われる。   
(この前に3文字の全角空白がある)}。

ひとつの改行は空白と同じに扱われる。
したがって、ソースファイルを見易くするために適宜改行を入れるとよい。
2つ以上連続した改行、つまり1つ以上の空行は段落の区切りと見なされる。

なお改ページには\verb|\pagebreak|を使用する。
例として、この文の直後に\verb|\pagebreak|を入れておく。


\pagebreak

\section{箇条書き}
\label{sec:kazyougaki}

\LaTeX としての記述のうち、
\verb+\begin{...}+ と \verb+\end{...}+ で
挟むものを環境と呼ぶ。
環境は箇条書きに限らない。

箇条書きのための環境として、itemize と enumerate がある。

たとえば、itemize 環境で次のように記述すると、
\ref{sec:kazyougaki}できじゅつしたように
\begin{quote}
\begin{verbatim}
\begin{itemize}
\item 情報科学類
\item 情報メディア創成学類
\item 知識情報・図書館学類
\end{itemize}
\end{verbatim}
\end{quote}
下のような箇条書きができる。

\begin{enumerate}
\item 情報科学類
\item 情報メディア創成学類
\item 知識情報・図書館学類
\end{enumerate}

enumerate 環境では下のように番号付けが行われる。

\begin{enumerate}
\item 生命環境学群
\item 理工学群
\item 情報学群
\end{enumerate}

箇条書きは入れ子にする(階層化する)ことも出来る。

\begin{enumerate}

\item 生命環境学群
\begin{enumerate}
\item 生物学類
\item 生物資源学類
\item 地球学類\label{item:earth}
\end{enumerate}

\item 理工学群

\item 情報学群\label{item:information}
\begin{itemize}
\item 情報科学類
\item[☆] 情報メディア創成学類
\item 知識情報・図書館学類
\end{itemize}

\end{enumerate}

enumerate環境に付けられる番号は参照できる。
階層化した方の箇条書きで、
情報学群には「\ref{item:information}」が、
地球学類には「\ref{item:earth}」が付けられている
(この番号は参照を使用している)。
情報メディア創成学類の記号だけ「☆」になっているが、
これは\verb|\item[☆]|のようにすることで箇条書きの記号を個別に変更している。

箇条書きには、ラベル名を指定できる description 環境もある。
箇条書きの記号やラベルをすべて個別に指定したい場合には description を使う方がよいだろう。
第\ref{sec:documentclass}節で示した、文書クラスの例には description 環境を使用している。


\end{document}
