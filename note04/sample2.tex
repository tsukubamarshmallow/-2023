\documentclass[fontsize=12pt,paper=a4]{jlreq}

\title{\LaTeX での数式の書き方}
\author{筑波大学 三末和男 (改訂:中井央)}
\date{2013年5月4日}

\begin{document}
\maketitle

\section{はじめに}

数式の記述は、\LaTeX の最も得意とする機能の1つである。
様々な数式をテキストだけで記述することができる
\footnote{電子メールや掲示板などプレーンテキストしか利用できない場合に、
数式を表現する際に \LaTeX の表記を流用することがある。
そのためにも覚えておくとよいであろう。}。
なお、本資料では様々な数式を例に挙げるが、
\LaTeX の機能説明のためのものであり、数式自体には特に意味はがない。



\section{数式の環境}\label{sec:math}

通常の文章に数式を入れるときは \$ と \$ の間にそれを挟む。例えば、
\begin{quote}
\begin{verbatim}
$f(x) = x + 1$
\end{verbatim}
\end{quote}
と記述すると、$f(x) = x + 1$ のように表示される。

数式を単独で、すなわち行を分けて記述するときには、displaymath 環境を用いるか、
代わりに \verb+\[+ と \verb+\]+ を使う。
数式を使うのが文中なのか、それとも単独行で使われるかによって、表示形式が変わるものもある。たとえば、単独行では、
\[
\sum_{i=1}^{n} i = \frac{n (n + 1)}{2}
\]
\noindent
のように表示される式が、文中では違う形式で表示される。

equation環境は displaymath 環境と似ているが数式番号が出るところが異なる。
equation環境で付与される数式番号は「\verb|\label{eq:no-1}|」のように
参照名を付けることができ、
「\verb|式(\ref{eq:no-1})|」のように記述することで「式(\ref{eq:no-1})」のように参照できる。

\begin{equation}
d_{k}(x_{1},y_{1},x_{2},y_{2})  = ((x_{1}-x_{2})^{k})+(y_{1}-y_{2})^{k})^{\frac{1}{k}}
\label{eq:no-1}
\end{equation}



\section{数式で用いる文字}

プレビューした結果を見ると分かるように、
数式ではアルファベットは斜字体で、数字はローマン体で書くのが普通である。
これらは数式の環境において自動的に指定される。

\subsection{括弧}

\[ (x), [x], \{x\}, \langle x \rangle, \lfloor x \rfloor, \lceil x \rceil \]

\subsection{ギリシア文字}
英字と同じ文字の場合には英字をそのまま使用する。一部の小文字には変体文字が用意されている。
\[ \alpha \beta \gamma \delta \epsilon \zeta \eta \theta \iota \kappa \lambda \mu \nu \xi \pi \rho \sigma \tau \upsilon \phi \chi \psi \omega \]
\[ \varepsilon \vartheta \varpi \varrho \varsigma \varphi \]
\[ \Gamma \Delta \Theta \Lambda \Xi \Pi \Sigma \Upsilon \Phi \Psi \Omega \]

\subsection{記号}
様々な演算子や関係記号が用意されている。関係記号は否定形も記述できる。
\[ \times, \div, \cap, \cup, \vee, \wedge, \setminus, \oplus, \otimes, \bigtriangleup, \angle \]
\[ \le, \ge, \subset, \supset, \subseteq, \supseteq, \in, \ni, \emptyset, \bar{A} \]
\[ \not\le, \not\ge, \not\subset, \not\supset, \notin, \not\ni \]
空集合の記号は$\emptyset$である。ギリシア文字の$\phi$とは異なるので注意すること。
様々な形の矢印も用意されている。

\[ \leftarrow, \Leftarrow, \rightarrow, \Rightarrow, \leftrightarrow, \Leftrightarrow \]

$i = 1, \ldots , n$や$s = 1 + \cdots + n$のような連続した点を表示する命令もある。ピリオドを連続させるわけではない。

\section{関数}

変数名等には斜体字が用いられるが、
関数名 ($\sin$ や $\cos$ など) はローマン体で書く習慣となっている。
そのため、専用の命令がある。
\verb|\sin|、\verb|\cos|、\verb|\lim|、\verb|\log|のように記述する。
そうすることで、$f(x) = \sin x + 2 \cos x$ や 
$\lim_{x \rightarrow \infty} \log_2 x$ のように表示される。

数式中で文字をローマン体で表示するには\verb|\mathrm{A}|のように記述する。

\section{べき乗、添字}

数式でよく使われる上付き文字(べき乗など)は\verb|$x^{10}$|のように記述する。
そうすると$x^{10}$のように表示される。
下付き文字(添字)は\verb|$x_{i}$|のように記述する。
そうすると$x_{i}$のように表示される。
上下に付く文字が1文字のときには\verb|$x^n$| 
や \verb|$x_n$|のように括弧を省略できる。
べき乗のべき乗、添字の添字、
添字付き変数のべき乗などいろいろな組合せも可能である。

\section{シグマ、積分、分数、平方根}

総和を表す($\sum_{n=0}^{10} a_n$)、総積を表す($\prod_{i=1}^{10} i$)、
積分 ($\int_0^\infty e^{-x} dx$) なども
簡単に記述することが出来る。これらも文中と単独行で形が変わる。
微分 ($f'(x), f''(x)$)や分数 ($\frac{x}{y}$)、
ルート ($\sqrt{2}$、$\sqrt[3]{5}$、$\sqrt{ \sqrt{x} + 1 }$)の記述も容易である。

\section{array環境}

equation環境やdisplaymath環境では数式はセンタリングされて通常中央に配置される。
しかし、複数の数式を列挙する場合には、等号の位置を揃えたいことがある。
そのような場合には、eqnarray環境を用いる。

\begin{eqnarray}
y & = & x^{2} + 2x + 1 \\
  & = & (x + 1) ^ {2} 
\end{eqnarray}

array環境を使用することで行列を記述することも出来る。

\begin{equation}
Q = \left[
\begin{array}{cc}
\cos\theta	& - \sin\theta		\\
\sin\theta	& \cos\theta
\end{array}
\right]
\end{equation}

条件付きの式は下のように表現できる。

\begin{equation}
|x| = \left\{
\begin{array}{rl}  % {rl}の代りに{ll}や{cc}なども試しなさい。
 x & \mbox{$x \ge 0$のとき} \\
-x & \mbox{それ以外}
\end{array}
\right.            % 右側の記述を忘れないこと。\right\}とすると括弧を閉じる。
\end{equation}

条件の部分は必ずしも数式ではないので\verb|\mbox{$x \ge 0$のとき}|のように\verb|\mbox|を使用して記述している。

$\theta = f(x) = \sin x $

\section{俺の練習}
定理3.11 Aをm*n行列としたときに

$L_{A}(x) = Ax $

で表現される写像$L_{A}: R^{m} \to R^{n}$は線形写像である。
\end{document}
